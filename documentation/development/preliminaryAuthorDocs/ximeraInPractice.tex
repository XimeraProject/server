\documentclass{amsart}
\usepackage{multicol}
\usepackage{kmath,kerkis}
\usepackage{multirow}
\usepackage{fancyvrb}


\begin{document}
\pagenumbering{gobble}
\title{XIMERA: Markup in Practice} %%%% CHANGE PER COURSE
\maketitle

\section{Writing Exercises}

A basic form of interactive is the ability to turn a static exercise
into an interactive exercise.

\begin{Verbatim}[frame=single,numbers=left]
\begin{exercise}
  Compute $2+2$
  \begin{answser}
    $17$
  \end{answer}
\end{exercise}
\end{Verbatim}

\begin{Verbatim}[frame=single,numbers=left]
\begin{exercise}
  Compute $2'+2'$
  \begin{answser}
    $17$ feet
  \end{answer}
\end{exercise}
\end{Verbatim}


\begin{Verbatim}[frame=single,numbers=left]
\begin{exercise}
  Compute $3\cdot 5$
  \begin{hint}
    $3\cdot 5$ represents $3$ groups of $5$
    \pause
    So we compute
    \[
    5+5+5
    \]
    \pause
    $=15$. 
  \end{hint}
  \begin{answser}
    $15$
  \end{answer}
\end{exercise}
\end{Verbatim}


\begin{Verbatim}[frame=single,numbers=left]
\begin{exercise}[\var{n}={integer from 0 to 10, 3}]
  Compute $2^\var{n}$.
  \begin{answser}
    $\evaluate{2^\var{n}}$ 
  \end{answer}
\end{exercise}
\end{Verbatim}


%% Here we have a ``random'' variable. integer vs real with a random
%% range given via ``from'' ``to'' and ``and''

\begin{Verbatim}[frame=single,numbers=left]
\begin{exercise}[\var{n}={integer from 1 to 10 and -1 to -10, 3}]
  Compute $1/\var{n}$.
  \begin{answser}
    $\evaluate{1/\var{n}}$ 
  \end{answer}
\end{exercise}
\end{Verbatim}



%% For this one we really need totally different prose for the latex
%% version and the text version. Of course if XIMERA works, then we
%% don't! The online version is good.

\begin{Verbatim}[frame=single,numbers=left]
\begin{exercise}
\begin{interactive}[limitsOracle.js]
  Find the value of each of the indicated limits by evaluating
  $f(x)$ at various values, or state that the limit does not
  exist (DNE).
\end{interactive}
\end{Verbatim}
\end{document}
